\documentclass[a4paper]{article}

\usepackage{fullpage} % Package to use full page
\usepackage{parskip} % Package to tweak paragraph skipping
\usepackage{tikz} % Package for drawing
\usepackage{amsmath}
\usepackage{hyperref}
\usepackage{amssymb}

\usepackage{enumitem}

\usepackage{outlines}
\usepackage{enumitem}
\setenumerate[1]{label=\Roman*.}
\setenumerate[2]{label=\Alph*.}
\setenumerate[3]{label=\roman*.}
\setenumerate[4]{label=\alph*.}


\title{COMP 339: Outline}
\author{Ling Tan}
\date{2018/09}

\begin{document}

\maketitle

\section*{1 Fundamental Principle of Counting}
\subsection*{1.1 The Rules of Sum and Product}
\subsection*{1.2 Permutations}
\subsection*{1.3 Combinations: The Binomial Theorem}
\subsection*{1.4 Combinations with Repetition}
\subsection*{1.5 The Catalan Numbers}

\section*{4 Properties of the Integers: Mathematical Induction}
\subsection*{4.1 The Well-Ordering Principle: Mathematical Induction}
\subsection*{4.2 Recursive Definitions}

\section*{8 The Principle of Inclusion and Exclusion}
\subsection*{8.1 The Principle of Inclusion and Exclusion}
\subsection*{8.2 Generalization of the Principle}

\section*{9 Generating Functions}
\subsection*{9.1 Introductory Examples}
\subsection*{9.2 Definition and Examples: Calculational Techniques}
\subsection*{9.3 Partitions of Integers}
\subsection*{9.4 The Exponential Generating Function}

\section*{10 Recurrence Relations}
\subsection*{10.4 The Method of Generating Functions}

\section*{11 An Introduction to Graph Theory}
\subsection*{11.1 Definitions and Examples}
\subsection*{11.2 Subgraphs, Complements, and Graph Isomorphism}
\subsection*{11.3 Vertex Degree: Euler Trails and Circuits}
\subsection*{11.4 Planar Graphs}
\subsection*{11.5 Hamilton Paths and Cycles}
\subsection*{11.6 Graph Coloring and Chromatic Polynomials}
\subsection*{11.7 Summary and Historical Review}

\section*{12 Trees}
\subsection*{12.1 Definitions, Properties, and Examples}

\section*{13 Optimization and Matching}
\subsection*{13.4 Matching Theory}


\begin{outline}[enumerate]
   \1 Level 1 
      \2 Level 2
         \3 Level 3
            \4 Level 4
\end{outline}
\end{document}