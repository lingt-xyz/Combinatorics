\documentclass[letter]{book}

\usepackage{fullpage} % Package to use full page
\usepackage{parskip} % Package to tweak paragraph skipping
\usepackage{tikz} % Package for drawing

\usepackage{amssymb}
\usepackage{amsmath}
\usepackage{amscd}

%https://en.wikibooks.org/wiki/LaTeX/Theorems
\usepackage{amsthm}
\theoremstyle{definition}
\newtheorem*{theorem*}{Theorem}
\newtheorem{theorem}{Theorem}[chapter]
\newtheorem{lemma}[theorem]{Lemma}
\newtheorem{corollary}[theorem]{Corollary}

\theoremstyle{definition}
\newtheorem*{mydef*}{Definition}
\newtheorem{mydef}{Definition}[chapter]
\newtheorem*{ex}{Example}

\newtheorem{prop}{Proposition}[chapter]
\theoremstyle{remark}
\newtheorem*{rem}{Remark}

\usepackage{enumitem}

\usepackage{tkz-graph}
\usepackage{float}

\usepackage{hyperref}

\usepackage{MnSymbol}

%https://tex.stackexchange.com/questions/229355/algorithm-algorithmic-algorithmicx-algorithm2e-algpseudocode-confused
\usepackage{algorithm}
\usepackage{algorithmicx}
\usepackage{algpseudocode}

\usepackage{graphicx}
\graphicspath{ {./resources/} }

%https://tex.stackexchange.com/questions/165021/fixing-the-location-of-the-appearance-in-algorithmicx-environment
\usepackage{float}% http://ctan.org/pkg/float

%https://tex.stackexchange.com/questions/25369/how-to-rotate-a-table
\usepackage[graphicx]{realboxes}

\begin{document}

\setcounter{chapter}{11}
\chapter{Trees}
\section{Definitions, Properties, and Examples}
\begin{mydef}
    Let $G=(V,E)$ be a loop-free undirected graph. The graph $G$ is called a tree if $G$ is connected and contains no cycles. A graph which has no loops, no cycles is called a forest.
\end{mydef}
\bigskip
\begin{mydef*}
    Spanning tree: for a connected graph $G$ is a connected subgraph (which is a tree) that contains all the vertices of $G$.
\end{mydef*}
\bigskip
\begin{mydef*}
    Spanning forest:
\end{mydef*}
\bigskip
\begin{theorem}
    If $a,b$ are distinct vertices in a tree $T=(V,E)$, then there is a unique path that connects these vertices.
    \begin{proof}
        By contradiction.\\
        Assume there are several paths from $a$ to $b$ in $T$.\\
        \begin{tikzpicture}[auto]
            \begin{scope}[every node/.style={circle,draw=black,minimum size=5mm}]
                \node (A) at (-5,0) {a};
                \node (B) at (-4,0) {};
                \node (C) at (-3,1) {$x_i$};
                \node (D) at (3,1) {$x_{k-1}$};
                \node (E) at (5,0) {$x_k,y_j$};
                \node (G) at (-3,-1) {$y_i$};
                \node (H) at (3,-1) {$y_{j-1}$};
                \node (K) at (7,0) {b};
            \end{scope}
            \begin{scope}[every edge/.style={draw=black}]
                \draw (A) edge node{} (B);
                \draw (B) edge node{} (C);
                \draw (B) edge node{} (G);
                \draw (C) edge[dashed] node{} (D);
                \draw (G) edge[dashed] node{} (H);
                \draw (D) edge node{} (E);
                \draw (H) edge node{} (E);
                \draw (E) edge node{} (K);
            \end{scope}
        \end{tikzpicture}
        \\
        Consider two of such paths: $a,x_1,x_2,\dots, x_r, b$ and $a, y_1,y_2,\ldots, y_s, b$. Find the smallest $i$ such that $x_i\neq y_i$, and smallest $j, j>i$ such that $y_j=x_k$. If not, the 2 paths must meet at $b$. In both cases $x_{i-1}x_i\cdots x_k (\text{or } b)y_{j-1}\cdots y_ix_{i-1}$ is a cycle.\\
        Or if $x_i\neq y_i$, then $x_1\cdots x_k (\text{or } b)y_{j-1}\cdots y_1$ is a cycle.\\
        Contradiction, since $T$ does not contain cycles.
    \end{proof}
\end{theorem}
\bigskip
\begin{theorem}
    If $G=(V,E)$ is an undirected graph, then $G$ is connected if and only if $G$ has a spanning tree (as a subgraph).
    \begin{proof}
        Should prove $\Leftarrow$ and $\Rightarrow$\\
        $\Leftarrow$ Assume $G$ has a spanning tree tree. By the previous result, we know there is a unique path between pair of vertices $a,b$ in the spanning tree. Then, by definition, $G$ is connected.\\
        $\Rightarrow$Assume $G$ is connected, we construct a subgraph $T$ of $G$ which contains all vertices of $G$, is connected, and contains no loops and cycles. First, remove all loops from $G$ to get $G_1$. If $G_1$ is not a tree, then it contains cycles. Find a cycle, remove an arbitrary edge from it to form $G_2$. Continue this process until no cycles remain in the resulting graph $G_t$. $G_t$ is a tree, and a spanning tree.
    \end{proof}
\end{theorem}
\bigskip
\begin{theorem}
    In every tree $T=(V,E), |V|=|E|+1$.
    \begin{proof}By induction on $|E|$.\\
        \begin{enumerate}
            \item Base: $|E|=0, |V|=1; |E|=1,|V|=2;|E|=2,|V|=3$.
            \item IH: Assume that in any tree with $|E|\leq k, |V|=|E|+1$.
            \item IS: Must prove that in any tree with $|E|=k+1,|V|=|E|+1$.
            \item Proof:Consider a tree $T$ with $|E|=k+1$. Choose an arbitrary edge $\{x,y\}$ and delete it. The resulting graph is a forest of exactly two trees $T_1=(V_1,E_1),T_2=(V_2,E_2),V_1\cup V_2=V,E_1\cup E_2=E\backslash\{x,y\}$. (If no, then there was another path between $x$ and $y$, which would create a cycle with $\{x,y\}$, which is impossible.) Alos, $|E_1|\leq k, |E_2|\leq k$. Then, IH applies to both trees $T_1,T_2\Rightarrow |V_1|=|E_1|+1,|V_2|=|E_2|+1$. $|V_1|+|V_2|=|V|=|E_1|+|E_2|+2=|E|-1+2=|E|+1$.
        \end{enumerate}
    \end{proof}
\end{theorem}
\bigskip
\begin{theorem}
    For every tree $T=(V,E)$, if $|V|\geq2$, then $T$ has at least two pendant vertices (degree is $1$).
    \begin{proof} By contradiction.\\
        Let $|V|=n\geq2$. Since $|E|=|V|-1=n-1$, then $2(n-1)=\sum_{v\in V}{deg(v)}$.\\
        Assume $T$ has $<2$ vertices of degree $1$. There are two cases:
        \begin{enumerate}
            \item $1$ vertex of degree $1$: other vertices have degree $\geq 2\Rightarrow \sum_{v\in V}{deg(v)}\geq 1+2(n-1)=2n-1$.
            \item no vertex of degree $1$
        \end{enumerate}
    \end{proof}
\end{theorem}
\bigskip
\begin{theorem}
    The following statements are equivalent for a loop-free undirected graph $G=(V,E)$.
    \begin{enumerate}[label=\alph*)]
        \item $G$ is a tree.
        \item $G$ is connected, but the removal of any edge from $G$ disconnects $G$ into two subgraphs that are trees.
        \item $G$ contains no cycles, and $|V|=|E|+1$.
        \item $G$ is connected, and $|V|=|E|+1$.
        \item $G$ contains no cycles, and if $a,b\in V$ with $\{a,b\}\notin E$, then the graph obtained by adding edge $\{a,b\}$ to $G$ has precisely one cycle.
        \item $\exists$ a unique path between any pair of vertices of $G$.
    \end{enumerate}
\end{theorem}
\begin{proof}
    Should prove: $a)\Rightarrow b)\Rightarrow c)\Rightarrow d)\Rightarrow e)\textcolor{red}{\Rightarrow a)}$
    \begin{enumerate}
        \item a) $\Rightarrow$ b) By contradiction.\\
            If $G$ is a tree then $G$ is connected. Let $e=\{a,b\}\in E$ be any edge of $G$.\\
            Assume $G-e$ is connected $\Rightarrow\exists$ at least two paths in $G$ from $a$ to $b$, contradiction. Then $G-e$ is disconnected and partition into two subsets.
            \begin{enumerate}
                \item vertex $a$, and all vertices of $G-e$ can be reached from $a$.
                \item vertex $b$, and all vertices of $G-e$ can be reached from $b$.
            \end{enumerate}
            Both components are trees (if no, a cycle in either component would also be in $G$).\\
            $\nexists$ a third class of vertices: vertices are not reachable from $a$ and not reachable form $b$ since $G$ is connected.
        \item b) $\Rightarrow$ c) By contradiction.\\
            Assume $G$ contains a cycle, let $e=\{a,b\}$ be an edge of the cycle. Remove $e$, then $G-e$ is still connected, contradiction, since from b) removal of $e$ should disconnected $G$. Thus, $G$ does not contain cycles, and $G$ is a tree, then $|V|=|E|+1$.
        \item c) $\Rightarrow$ d) By contradiction.\\
            Assume $G$ is not connected, let $G_1,G_2,\ldots, G_r$ be the connected components of $G$; that is, $G$ is a forest of $r$ trees. Select a vertex $v_i\in V_{G_i},i=1,2,\ldots, r$ and add the $r-1$ edges: $\{v_1,v_2\},\{v_2,v_3\},\ldots, \{v_{r-1},v_r\}$ to $G$ form a graph $G'=(V,E')$, which is a tree, because edges $\{v_1,v_2\},\{v_2,v_3\},\ldots, \{v_{r-1},v_r\}$ do not create cycles. Clearly, $|E'|=|E|+r-1$. Since $G'$ is a tree, then $|V|=|E'|+1$, but $|V|=|E|+1\Rightarrow |E|=|E'|=|E|+r-1\Rightarrow r-1=0\Rightarrow r=1\Rightarrow G$ is connected.
        \item d) $\Rightarrow$ e)
            \begin{enumerate}
                \item $G$ contains no cycle: If $G$ contains a cycle, then find a cycle in $G$ and remove one of its edges. Continue this process until get a graph $G'$ without cycles. $G'$ is a tree, assume we deleted $k$ edges from $G$ to obtain $G'=(V,E'),|E'|=|E|-k$. Since $G'$ is a tree, then $|V|=|E'|+1=|E|-k+1$. Also, from d) $|V|=|E|+1\Rightarrow |E|+1=|E|-k+1\Rightarrow = k\Rightarrow$ there was no cycle in $G$.
                \item Precisely $1$ cycle. Add one edge $\{a,b\}$ to $G$. Assume $2$ or more cycles created in $G'=G+\{a,b\}$. Take one of the cycles, delete an edge. Take the second cycle, delete an edge, then we obtain a tree $G''=(V,E''), |E''|=|E|+2-1=|E|-1$. Since $G''$ is a tree, then $|V|=|E''|+1=(|E|-1)+1=|E|$, contradiction, because $|V|=|E|+1$.
            \end{enumerate}
        \item e) $\Rightarrow$ a) By contradiction.\\
            Assume $G$ is not connected, that is, $G$ is a forest of $k$ trees of $G_i,i=1,2,\ldots, k$. Take $v_i\in V_{G_i}$, connect $\{v_1,v_2\}$ to create a graph $G'=(V,E'),E'=E\cup \{v_1,v_2\}$. By e), $G'$ must have precisely $1$ cycle, and $\{v_1,v_2\}$ is part of this cycle. Then, there was another path from $v_1$ to $v_2$ in $G$. Thus $G_1$ and $G_2$ are connected, contradiction. Therefore, $G$ is not a forest, and is a tree.
    \end{enumerate}
\end{proof}
\section{Rooted Trees}
\begin{mydef}
    If $G$ is a directed graph, the $G$ is called a directed tree if the undirected graph associated with $G$ is a tree. When $G$ is a directed tree, $G$ is called 
\end{mydef}
\end{document}