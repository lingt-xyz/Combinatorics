\documentclass[a4paper]{article}

\usepackage{fullpage} % Package to use full page
\usepackage{parskip} % Package to tweak paragraph skipping
\usepackage{tikz} % Package for drawing
\usepackage{amsmath}
\usepackage{hyperref}
\usepackage{amssymb}

\usepackage{enumitem}

\title{9 Generating Functions}
\author{Ling Tan}
\date{2018/09}

\begin{document}

\maketitle

\section*{9.1 Introductory Examples}
Mildred buys 12 oranges for her children, Grace, Mary, and Frank. In how many ways can she distribute the oranges so that Grace gets at least four, and Mary and Frank get at least two, but Frank gets no more than five?
$$c_1+c_2+c_3=12\text{, where } 4\leq c_1, 2\leq c_2, 2\leq c_3\leq 5$$
The coefficient of $x^{12}$ in 
$$f(x)=(x^4+x^5+x^6+x^7+x^8)(x^2+x^3+x^4+x^5+x^6)(x^2+x^3+x^4+x^5)$$
counts the number of distributions that we seek.

\section*{9.2 Definition and Examples: Calculational Techniques}
\textcolor{blue}{Definition 9.1} Let $a_0,a_1,a_2,\ldots$ be a sequence of real numbers. The function
$$f(x)=a_0+a_1x+a_2x^2+\cdots=\sum_{i=0}^{\infty}{a_ix^i}$$
is called the generating function for the given sequence.
\subsection*{Examples}
\begin{enumerate}
    \item $(1+x)^n$
    \begin{align*}
    &=\binom{n}{0}+\binom{n}{1}x+\binom{n}{2}x^2+\binom{n}{3}x^3+\cdots+\binom{n}{n}x^n\\
    &\text{ is the generating function for the sequence } \binom{n}{0},\binom{n}{1},\binom{n}{2},\binom{n}{3},\ldots,\binom{n}{n}
    \end{align*}
    \item $\frac{(1-x^{n+1})}{(1-x)}$
    \begin{align*}
        (1-x^{n+1})&=(1-x)(1+x+x^2+x^3+\cdots+x^n)\\
        \Rightarrow \frac{(1-x^{n+1})}{(1-x)}&=(1+x+x^2+x^3+\cdots+x^n)\\
        &\text{ is the generating function for the sequence } 1,1,1,\cdots,1,0,0,0,\cdots
    \end{align*}
    
    \item $\frac{1}{(1-x)}$
    \begin{align*}
        1&=(1-x)(1+x+x^2+x^3+\cdots)\\
        \Rightarrow \frac{1}{(1-x)}&=(1+x+x^2+x^3+\cdots)\\
        &\text{ is the generating function for the sequence } 1,1,1,\cdots
    \end{align*}
    
    \item $\frac{1}{(1-x^3)}$
    \begin{align*}
        &=(1+x^3+x^6+x^9+\cdots)\\
        &\text{ is the generating function for the sequence } 1,0,0,1,0,0,1,\cdots
    \end{align*}
    
    \item $\frac{1}{(1+x^2)}$
    \begin{align*}
        &=\frac{1}{\bigl(1-(-x^2)\bigr)}\\
        &=(1-x^2+x^4-x^6+\cdots)\\
        &\text{ is the generating function for the sequence } 1,0,-1,0, 1,0,-1,\cdots\\
        &\text{ with } a_n= 
            \begin{cases}
            1,  & \text{if $n\equiv 0\pmod  4$} \\
            -1,  & \text{if $n\equiv 2\pmod  4$} \\
            0, & \text{if $n$ is odd}
            \end{cases}
    \end{align*}
    
    \item $\frac{x}{(1+x^2)}$
    \begin{align*}
        &=x-x^3+x^5-x^7+\cdots\\
        &\text{ with } a_n= 
            \begin{cases}
            0,  & \text{if $n$ is even} \\
            1,  & \text{if $n\equiv 1\pmod  4$} \\
            -1, & \text{if $n\equiv 3\pmod  4$} \\
            \end{cases}
    \end{align*}
    
    \item $\frac{(1+x)^2}{(1+x^2)(1-2x)}$
    \begin{align*}
        &=\frac{Ax+B}{1+x^2}+\frac{C}{1-2x}\\
        &=\frac{1}{5}\Bigl(\frac{2x}{1+x^2}-\frac{4}{1+x^2}+\frac{9}{1-2x}\Bigr)\\
        \text{ with } a_n&=\frac{1}{5}\Biggl(2\cdot
            \begin{cases}
            0,  & \text{if $n$ is even} \\
            1,  & \text{if $n\equiv 1\pmod  4$} \\
            -1, & \text{if $n\equiv 3\pmod  4$} \\
            \end{cases}\\
        &
            -4\cdot
            \begin{cases}
            0,  & \text{if $n$ is odd} \\
            1,  & \text{if $n\equiv 0\pmod  4$} \\
            -1, & \text{if $n\equiv 2\pmod  4$} \\
            \end{cases}\\
        & 
            +9\cdot 2^n 
            \Biggr)\\
        \Rightarrow a_7&=\frac{1}{5}\bigl(2\cdot(-1)-4\cdot 0 + 9\cdot 2^7\bigr)=230
    \end{align*}
    
    \item $\frac{1}{(1-2x)}$
    \begin{align*}
        &=(1+2x+4x^2+8x^3+\cdots)\\
        &\text{ is the generating function for the sequence } 1,2,4,8,\cdots\\
        &\text{ with } a_n=2^n
    \end{align*}
    
    \item $\frac{x}{(1-x)}$
    \begin{align*}
        &=x\cdot\frac{1}{1-x}\\
        &=x\cdot(1+x+x^2+x^3+\cdots)\\
        &=(x+x^2+x^3+\cdots)\\
        &\text{ is the generating function for the sequence } 0,1,1,1,\cdots
    \end{align*}
    
    \item $\frac{1}{(1-x)^2}$
    \begin{align*}
        \frac{1}{1-x}&=1+x+x^2+x^3+\cdots\\
        \Rightarrow \frac{1}{(1-x)^2}&=1+2x+3x^2+\cdots\qquad\text{(differentiate)}\\
        &\text{ is the generating function for the sequence } 1,2,3,\cdots
    \end{align*}
    
    \item $\frac{1+x}{(1-x)^3}$
    \begin{align*}
        \frac{x}{(1-x)^2}&=x+2x^2+3x^3+4x^4+\cdots\\
        \Rightarrow \frac{1+x}{(1-x)^3}&=1+4x+9x^2+16x^3+\cdots\qquad\text{(differentiate)}\\
        &\text{ is the generating function for the sequence } 1,4,9,16,\cdots
    \end{align*}
    
    \item $??$
    \begin{align*}
        \Rightarrow \qquad\text{(differentiate)}\\
        &\text{ is the generating function for the sequence } 1^3,2^3,3^3,4^3,\cdots
    \end{align*}
\end{enumerate}

\textcolor{blue}{Definition} With $n,r\in\mathbb{Z}^+$ and $n\geq r>0$, we have
$$\binom{n}{r}=\frac{n!}{r!(n-r)!}=\frac{n(n-1)(n-2)\cdots(n-r+1)}{r!}$$
If $n\in\mathbb{R}$, we use
$$\frac{n(n-1)(n-2)\cdots(n-r+1)}{r!}$$
as the definition of $\binom{n}{r}$. Then, for example, if $n\in\mathbb{Z}^+$, we have
\begin{align*}
\binom{-n}{r}&=\frac{(-n)(-n-1)(-n-2)\cdots(-n-r+1)}{r!}\\
&=\frac{(-1)^r(n)(n+1)(n+2)\cdots(n+r-1)}{r!}\\
&=\frac{(-1)^r(n+r-1)!}{(n-1)!r!}\\
&=(-1)^r\binom{n+r-1}{r}
\end{align*}
Finally, for each real number $n$, we define $\binom{n}{0}=1$

\subsection*{Examples}
\begin{enumerate}
    \item For $n\in\mathbb{Z}^+$, the Maclaurin series expansion for $(1+x)^{-n}$ is given by
    \begin{align*}
        (1+x)^{-n}&=1+(-n)x+(-n)(-n-1)\frac{x^2}{2!}+(-n)(-n-1)(-n-2)\frac{x^3}{3!}+\cdots\\
        &=1+\sum_{r=1}^{\infty}{\frac{(-n)(-n-1)(-n-2)\cdots(-n-r+1)}{r!}x^r}\\
        &=\sum_{r=0}^{\infty}{(-1)^r\binom{n+r-1}{r} x^r}=\sum_{r=0}^{\infty}{\binom{-n}{r}x^r}\\
        \Rightarrow(1+x)^{-n}&=\binom{-n}{0}+\binom{-n}{1}x+\binom{-n}{2}x^2+\cdots
    \end{align*}
    \item How many 4-element subset from $\{1,2,3,\ldots,15\}$ with no consecutive integers?\\
    Let $\{a_1,a_2,a_3,a_4\}, a_1<a_2<a_3<a_4$ be such a subset.\\
    Define $C_1=a_1-1,C_2=a_2-a_1,C_3=a_3-a_2,C_4=a_4-a_3,C_5=15-a_4$\\
    $\Rightarrow C_1+C_2+C_3+C_4+C_5=14;C_1,C_5\geq 0; C_2,C_3,C_4>1$\\
    $\Rightarrow$ Find $C_i \equiv$ Find $a_i \equiv$ Find the coefficient of $x^{14}$
    \begin{align*}
    f(x)&=(1+x+x^2+\cdots+x^{14})^2(x^2+x^3+\cdots+x^{14})^3\\
    \equiv g(x)&=(1+x+x^2+\cdots)^2(x^2+x^3+\cdots)^3\\
    &=x^6(1+x+x^2+\cdots)^5\\
    &=\frac{x^6}{(1-x)^5}=x^6\sum_{r=0}^{\infty}{\binom{5+r-1}{r}x^r}\\
    \frac{x^{14}}{x^6}&=x^8\Rightarrow r=8\\
    \Rightarrow&\binom{5+r-1}{r}=\binom{12}{8}=495
    \end{align*}
    \item \textcolor{red}{Example 9.14: Determine the coefficient of $x^{24}$ in $f(x)=(x^3+x^4+\cdots+x^8)^4$}
    \begin{align*}
        f(x)&=(x^3+x^4+\cdots+x^8)^4\\
        &=x^{12}(1+x+x^2+\cdots+x^5)^4\\
        &=x^{12}\Bigl(\frac{1-x^6}{1-x}\Bigr)^4
    \end{align*}
    The answer is the coefficient of $x^{12}$ in $(1-x^6)^4\cdot (1-x)^{-4}$.
    \begin{align*}
        (1-x^6)^4\cdot (1-x)^{-4}&=\Bigg[1-\binom{4}{1}x^6+\binom{4}{2}x^{12}-\binom{4}{3}x^{18}+x^{24}\Bigg]\\
        &\cdot\Bigg[\binom{-4}{0}+\binom{-4}{1}(-x)+\binom{-4}{2}(-x)^2+\cdots\Bigg]
    \end{align*}
    The coefficient of $x^{12}$ is
    $$
    \Bigg[\binom{-4}{12}(-1)^{12}-\binom{4}{1}\binom{-4}{6}(-1)^6+\binom{4}{2}\binom{-4}{0}\Bigg]=\Bigg[\binom{15}{12}-\binom{4}{1}\binom{9}{6}+\binom{4}{2}\Bigg]=125.
    $$
    \item Example 9.13: The special compositions for a number that read the same left to right as right to left are the palindromes of that number.\\
    For $n=1$ there is one palindrome.\\
    If $n=2k+1$, for $k\in\mathbb{Z}^+$, then there is one palindrome with center summand $n$. For $1\leq t\leq k$, there are $2^{t-1}$ palindromes of $n$ with center summand $n-2t$. Hence the total number of palindromes of $n$ is $1+(1+2^1+2^2+\cdots+2^{k-1})=1+(2^k-1)=2^k=2^{(n-1)/2}$.\\
    If $n=2k$, for $k\in\mathbb{Z}^+$. Here there is one palindrome with center summand $n$ and, for $1\leq s\leq k-1$, there are $2^{s-1}$ palindromes of $n$ with center summand $n-2s$. Id addition, there are $2^{k-1}$ palindromes where a plus sign is at the center. In total, $n$ has $1+(1+2^1+2^2+\cdots+2^{k-1})=1+(2^k-1)=2^k=2^{n/2}$ palindromes.
    $$
    n \text{ has } 2^{\lfloor n/2\rfloor} \text{ palindromes.}
    $$
\end{enumerate}

\section*{9.3 Partitions of Integers}
In number theory, we are confronted with partitioning a positive integer $n$ into positive summands and seeking the number of such partitions, without regard to order. This number is denoted by $p(n)$. For example,
\begin{align*}
p(1) = 1:\quad 1&\\
p(2) = 2:\quad 2& = 1+1\\
p(3) = 3:\quad 3& = 2+1 = 1+1+1\\
p(4) = 5:\quad 4&=3+1=2+2=2+1+1=1+1+1+1\\
p(5) = 7:\quad 5&=4+1=3+2=3+1+1=2+2+1=2+1+1+1=1+1+1+1+1
\end{align*}
If $n\in \mathbb{Z}^+$, the number of $1$'s we can use is $0$ or $1$ or $2$ or $\dots$. The power series $1+x+x^2+x^3+\cdots$ keeps account of this. In like manner, $1+x^2+x^4+x^6+\cdots$ keeps track of the number of $2$'s in the partition of $n$, while $1+x^3+x^6+\cdots$ accounts for the number of $3$'s.
$$f(x)=\frac{1}{(1-x)}\frac{1}{(1-x^2)}\frac{1}{(1-x^3)}\cdots=\prod_{i=1}^{\infty}{\frac{1}{(1-x^i)}}$$ generates the sequence $p(0),p(1),p(2),p(3),\ldots$, where we define $p(0)=1$.\\
\\
\textcolor{red}{e.g.}: In order to determine $p(10)$, for instance, we want the coefficient of $x^{10}$ in $f(x)=(1+x+x^2+x^3+\cdots)(1+x^2+x^4+\cdots)(1+x^3+x^6+\cdots)\cdots(1+x^{10}+x^{20}+x^{30}+\cdots)\cdots$.\\
$x^{10}=x^2\cdot x^2\cdot x^6\Rightarrow p(10)=(1+1)+(2)+(3+3)$

\section*{9.4 The Exponential Generating Function}
\begin{itemize}
    \item Ordinary generating function: Order is not relevant; useful in selection problems
    $$G(a_n;x)=\sum_{n=0}^{\infty}{a_nx^n}$$
    \item Exponential generating function: Order is crucial
    $$EG(a_n;x)=\sum_{n=0}^{\infty}{a_n\frac{x^n}{n!}}$$
\end{itemize}
\textcolor{blue}{Definition}: For a sequence $a_0, a_1, a_3, \dots$ of real numbers, $$f(x)=a_0+a_1x+a_2\frac{x^2}{2!}+a_3\frac{x^3}{3!}+\dots=\sum_{i=0}^{\infty}{a_i\frac{x^i}{i!}},$$ 
is called the exponential generating function for the given sequence.

\subsection*{Examples}
\begin{enumerate}
    \item Binomial theorem
    \begin{align*}
    (1+x)^n&=\binom{n}{0}+\binom{n}{1}x+\binom{n}{2}x^2+\binom{n}{3}x^3+\cdots+\binom{n}{n}x^n\\
    &\text{is the ordinary generating function for the sequence }\binom{n}{0},\binom{n}{1},\dots,\binom{n}{n}\\
    (1+x)^n&=p(n,0)+p(n,1)x+p(n,2)\frac{x^2}{2!}+p(n,3)\frac{x^3}{3!}+\cdots+p(n,n)\frac{x^n}{n!}\\
    &\text{is the exponential generating function for the sequence }p(n,0),p(n,1),p(n,2),\dots, p(n,n)
    \end{align*}

\item Maclaurin series expansion of $e^x$
\begin{align*}
e^x&=1+x+\frac{x^2}{2!}+\frac{x^3}{3!}+\cdots=\sum_{i=0}^{\infty}{\frac{x^i}{i!}}\\
&\text{is the ordinary generating function for the sequence }1,1,\frac{1}{2!},\frac{1}{3!},\cdots\\
&\text{is the exponential generating function for the sequence }1,1,1,\dots\\
\end{align*}

\item $e^{-x}$
\begin{align*}
e^x&=1-x+\frac{x^2}{2!}-\frac{x^3}{3!}+\cdots\\
&\text{is the ordinary generating function for the sequence }1,-1,\frac{1}{2!},-\frac{1}{3!},\cdots\\
&\text{is the exponential generating function for the sequence }1,-1,1,-1,\dots\\
\end{align*}

\item $\frac{1}{2}(e^x+e^{-x})$
\begin{align*}
\frac{1}{2}(e^x+e^{-x})&=1+\frac{x^2}{2!}+\frac{x^4}{4!}+\cdots\\
&\text{is the exponential generating function for the sequence }1,0,1,0,\dots\\
\end{align*}

\item $\frac{1}{2}(e^x-e^{-x})$
\begin{align*}
\frac{1}{2}(e^x-e^{-x})&=x+\frac{x^3}{3!}+\frac{x^5}{5!}+\cdots\\
&\text{is the exponential generating function for the sequence }0,1,0,1,\dots\\
\end{align*}

\item Example 9.28: A ship carries 48 flags, 12 each of the colors read, white, blue, and black. Twelve of these flags are placed on a vertical pole in order to communicate a signal to other ships.
\begin{enumerate}[label=(\alph*)]
\item How many of these signals use an even number of blue flags and an odd number of black flags?\\
The exponential generating function
$$
f(x)=\Bigl(1+x+\frac{x^2}{2!}+\frac{x^3}{3!}+\cdots\Bigr)^2\Bigl(1+\frac{x^2}{2!}+\frac{x^4}{4!}+\cdots\Bigr)\Bigl(x+\frac{x^3}{3!}+\frac{x^5}{5!}+\cdots\Bigr)
$$
The last two factors in $f(x)$ restrict the signals to an even number of blue flags and an odd number of black flags, respectively.
\begin{align*}
f(x)=&\Bigl(1+x+\frac{x^2}{2!}+\frac{x^3}{3!}+\cdots\Bigr)^2\Bigl(1+\frac{x^2}{2!}+\frac{x^4}{4!}+\cdots\Bigr)\Bigl(x+\frac{x^3}{3!}+\frac{x^5}{5!}+\cdots\Bigr)\\
=&(e^x)^2\cdot\frac{1}{2}(e^x+e^{-x})\cdot\frac{1}{2}(e^x-e^{-x})\\
=&\frac{1}{4}(e^{2x})\cdot(e^{2x}-e^{-2x})\\
=&\frac{1}{4}(e^{4x}-1)\\
=&\frac{1}{4}\Bigl(\sum_{i=0}^{\infty}{\frac{(4x)^i}{i!}}-1\Bigr)\\
=&\frac{1}{4}\sum_{i=1}^{\infty}{\frac{(4x)^i}{i!}}\\
\end{align*}
consider all such signals. The coefficient of ${x^{12}}/{12!}$ yields $(1/4)(4^{12})=4^{11}$ signals.
\end{enumerate}
\end{enumerate}
\end{document}