\documentclass[a4paper]{article}

\usepackage{fullpage} % Package to use full page
\usepackage{parskip} % Package to tweak paragraph skipping
\usepackage{tikz} % Package for drawing
\usepackage{amsmath}
\usepackage{hyperref}
\usepackage{amssymb}
\usepackage{enumitem}

\title{COMP 339 Assignment 1}
\author{Ling Tan}
\date{2018/09/17}

\begin{document}

\maketitle

\section*{Exercises 1.1 and 1.2}
\begin{description}
\item[10]
Pamela has $15$ different books. In how many ways can she place her books on two shelves so that there is at least one book on each shelf?\\
$15! \cdot 14 = 1.8307441\cdot10^{13}$
\item[38]
A committee of $15$, nine women and six men is to be seated at a circular table. In how many ways can the seats be assigned so that no two men are seated next to each other?\\
$(9!/9) \cdot 9 \cdot 8 \cdot 7 \cdot 6 \cdot 5 \cdot 4 = 2438553600 $
\end{description}
\section*{Exercises 1.3}
\begin{description}
\item[8]
In how many ways can a gambler draw five card from a standard deck and get
    \begin{enumerate}[label=(\alph*)]
      \item a flush (any five cards all of the same suit)\\
      $ \binom{13}{5}\cdot\binom{4}{1}=1287\cdot 4=5148 $
      \item four aces\\
      $ \binom{4}{4}\cdot\binom{48}{1}=1\cdot 48=48$
      \item four of a kind (same number)\\
      $\binom{13}{1}\cdot\binom{48}{1}=13\cdot 48=624$
      \item three aces and two jacks\\
      $ \binom{4}{3}\cdot\binom{4}{2}=4\cdot6=24$
      \item three aces and a pair\\
      $\binom{4}{3}\cdot\binom{12}{1}\cdot\binom{4}{2}=4\cdot12\cdot6=288$
      \item a full house (three of a kind and a pair)\\
      $\binom{13}{1}\cdot\binom{4}{3}\cdot\binom{12}{1}\cdot\binom{4}{2}=3744$
      \item three of a kind (the other two are not of a kind)\\
      $\binom{13}{1}\cdot\binom{4}{3}\cdot\binom{12}{2}\cdot\binom{4}{1}\binom{4}{1}=13\cdot4\cdot66\cdot 4\cdot 4=54912$
      \item two pairs\\
      $\bigg(\Big(\binom{52}{1}\cdot\binom{3}{1}/2\Big)\cdot\Big(\binom{48}{1}\cdot\binom{3}{1}/2\Big)\bigg)/2\cdot\binom{44}{1}=123552=\binom{13}{2}\binom{4}{2}\binom{4}{2}\binom{44}{1}$
    \end{enumerate}

\item[18]
For the strings (consists of the symbols $0,1,2$) of length $10$, how many have
    \begin{enumerate}[label=(\alph*)]
      \item four 0's, three 1's, and three 2's\\
      $ \binom{10}{4}\cdot\binom{6}{3} =4200=\frac{10!}{4!3!3!}$
      \item at least eight 1's\\
      $\binom{10}{8}\cdot2\cdot2+\binom{10}{9}\cdot2+\binom{10}{10}=45\cdot4+10\cdot2+1=201$
      \item weight 4\\
      $\binom{10}{4}+\binom{10}{2}\cdot\binom{8}{1}+\binom{10}{2}=210+ 45\cdot 8+45=615$
    \end{enumerate}

\item[26]
Find the coefficient of $w^2x^2y^2z^2$ in the expansion of
    \begin{enumerate}[label=\alph*)]
      \item $(w+x+y+z+1)^{10}$\\
      $\binom{10}{2}\cdot\binom{8}{2}\cdot\binom{6}{2}\cdot\binom{4}{2}\cdot\binom{2}{2}=45\cdot 28\cdot 15\cdot 6\cdot 1=113400$
      \item $(2w-x+3y+z-2)^{12}$\\
      $\binom{12}{2}\cdot2^2\cdot\binom{10}{2}\cdot(-1)^2\cdot\binom{8}{2}\cdot3^2\cdot\binom{6}{2}\cdot\binom{4}{4}\cdot(-2)^4=66\cdot4\cdot 45\cdot 1\cdot 28\cdot 9\cdot 15\cdot 1\cdot 16=718502400$
      \item $(v+w-2x+y+5z+3)^{12}$\\
      $\binom{12}{2}\cdot\binom{10}{2}\cdot(-2)^2\binom{8}{2}\binom{6}{2}\cdot5^2\cdot\binom{4}{4}\cdot3^4=66\cdot 45\cdot 4\cdot 28\cdot 15\cdot 25\cdot 1 \cdot 81=10103940000$
    \end{enumerate}

\item[28]
For any positive integer $n$ determine
    \begin{enumerate}[label=\alph*)]
      \item $\sum_{i=0}^n\frac{1}{i!(n-i)!}$
        \begin{align*}
            n!\cdot \sum_{i=0}^n\frac{1}{i!(n-i)!}  & = \sum_{i=0}^n\frac{n!}{i!(n-i)!} \\
            & = \sum_{i=0}^n\binom{n}{i} \\
            & = \sum_{i=0}^n\binom{n}{i}1^i1^{n-i} \\
            & = 2^n\\
            \Rightarrow\sum_{i=0}^n\frac{1}{i!(n-i)!}& = 2^n/n!
        \end{align*}
      \item $\sum_{i=0}^n\frac{(-1)^i}{i!(n-i)!}$
        \begin{align*}
            n!\cdot \sum_{i=0}^n\frac{(-1)^i}{i!(n-i)!}  & = \sum_{i=0}^n\frac{(-1)^i\cdot n!}{i!(n-i)!} \\
            & = \sum_{i=0}^n (-1)^i\cdot\frac{n!}{i!(n-i)!} \\
            & = \sum_{i=0}^n\binom{n}{i}(-1)^i1^{n-i} \\
            & = 0\\
            \Rightarrow\sum_{i=0}^n\frac{(-1)^i}{i!(n-i)!}& = 0/n!=0
        \end{align*}
    \end{enumerate}
\end{description}
\section*{Exercises 1.4}
\begin{description}
\item[12]
Determine the number of integer solutions for
\begin{equation*}
    x_1+x_2+x_3+x_4+x_5<40,
\end{equation*}
where
    \begin{enumerate}[label=\alph*)]
      \item $x_i \ge 0 $\\
      $x_1+x_2+x_3+x_4+x_5+x_6=40,x_6>=1$\\
      $\Rightarrow x_1+x_2+x_3+x_4+x_5+x_6=39,x_i\ge 0$\\
      $\binom{39+6-1}{39}=\binom{44}{39}=1086008 $
      \item $x_i \ge -3 $\\
      $x_1+x_2+x_3+x_4+x_5<55,x_i\ge 0$\\
      $x_1+x_2+x_3+x_4+x_5+x_6=55,x_6>=1$\\
      $x_1+x_2+x_3+x_4+x_5+x_6=54,x_i>=0$\\
      $\binom{39+3\cdot5+6-1}{39+3\cdot5}=\binom{59}{54}=5006386 $
    \end{enumerate}

\item[16]
For which positive integer $n$ will be the equations
\begin{enumerate}[label=(\arabic*)]
    \item $x_1+x_2+x_3+\cdots+x_{19}=n$, and
    \item $y_1+y_2+y_3+\cdots+y_{64}=n$.
\end{enumerate}
have the same number of positive integer solutions?
    \begin{equation*}
    \begin{cases}
    x_1+x_2+x_3+\cdots+x_{19}=n\\
    y_1+y_2+y_3+\cdots+y_{64}=n
    \end{cases}
    \Rightarrow
    \begin{cases}
    x_1+x_2+x_3+\cdots+x_{19}=n-19\\
    y_1+y_2+y_3+\cdots+y_{64}=n-64
    \end{cases}
    \end{equation*}
    \begin{align*}
        \binom{19+(n-19)-1}{n-19} & = \binom{64+(n-64)-1}{n-64} \\
        \binom{n-1}{n-19} & = \binom{n-1}{n-64} \\
        \binom{n-1}{18} & = \binom{n-1}{63} \\
        \Rightarrow n-1 & = 18+63=81\Rightarrow n=81+1=82
    \end{align*}
\item[18]
How many integer solutions are there to the pair of equations $x_1+x_2+x_3+\cdots x_7=37,x_1+x_2+x_3=6$?
    \begin{enumerate}[label=\alph*)]
    \item non-negative
    \begin{equation*}
    \begin{cases}
    x_1+x_2+x_3+\cdots x_7&=37\\
    x_1+x_2+x_3&=6
    \end{cases}
    \Rightarrow
    \begin{cases}
    x_4+x_5+x_6+x_{7}&=37-6=31\\
    x_1+x_2+x_3&=6
    \end{cases}
    \end{equation*}
      $ \binom{4+31-1}{31}\cdot\binom{3+6-1}{6}=\binom{34}{31}\cdot\binom{8}{6}=5984\cdot 28=167552$
      \item $x_1,x_2,x_3>0$
     \begin{equation*}
    \begin{cases}
    x_4+x_5+x_6+x_{7}&=31\\
    x_1+x_2+x_3&=6
    \end{cases}
    \Rightarrow
    \begin{cases}
    x_4+x_5+x_6+x_7&=31\\
    x_1+x_2+x_3&=3
    \end{cases}
    \end{equation*}
      $\binom{4+31-1}{31}\cdot\binom{3+3-1}{3}=\binom{34}{31}\cdot\binom{5}{3}=5984\cdot 10=59840$
    \end{enumerate}
\end{description}
\section*{Exercises 4.2, P241}
\begin{description}


\item[10] Prove that $| x_1+x_2+\dots+x_{n-1}+x_n|\leq|x_1|+|x_2|+\dots+|x_{n-1}|+|x_n|,n\in\mathbb{Z}^+,n\ge2 $\\
\textbf{Base:} If $n=2, |x_1+x_2|\leq|x_1|+|x_2|$ \hfill Given\\
\textbf{Inductive Hypothesis:} Assume $|x_1+x_2+\dots+x_n|\leq|x_1|+|x_2|+\dots+|x_k|,k\in\mathbb{Z}^+,k=2,3,\dots,n$\\
\textbf{Inductive Step:} Must prove $|x_1+x_2+\dots+x_n+x_{n+1}|\leq|x_1|+|x_2|+\dots+|x_n|+|x_{n+1}|$\\
\textbf{Proof:}\\
$|(x_1+x_2+\dots+x_k)+x_{k+1}|\leq|x_1+x_2+\dots+x_k|+|x_{k+1}| $ \hfill Base\\
$|x_1+x_2+\dots+x_n|\leq|x_1|+|x_2|+\dots+|x_n|$ \hfill IH\\
$\Rightarrow |(x_1+x_2+\dots+x_n)+x_{n+1}|\leq(|x_1|+|x_2|+\dots+|x_n|)+|x_{n+1}| $ \hfill Base \& IH\\
$\Rightarrow |x_1+x_2+\dots+x_n+x_{n+1}|\leq|x_1|+|x_2|+\dots+|x_n|+|x_{n+1}| $ \hfill \\
\textbf{Conclusion: }$| x_1+x_2+\dots+x_{n-1}+x_n|\leq|x_1|+|x_2|+\dots+|x_{n-1}|+|x_n|,n\in\mathbb{Z}^+,n\ge2 $

\item[12 Prove by induction] Prove that $F_0+F_1+F_2+\dots+F_n=F_{n+2}-1,n\ge0 $\\
\textbf{Base:} \\
If $n=0, F_0=\sum_{i=0}^0 F_i=F_0=0, F_2-1=1-1=0\Rightarrow LHS=RHS$ \hfill $F_0=0, F_2=1$\\
If $n=1, F_1=\sum_{i=0}^1 F_i=F_0+F_1=1, F_3-1=2-1=1\Rightarrow LHS=RHS$ \hfill $F_0=0, F_1=1,F_3=2$\\
\textbf{Inductive Hypothesis:} Assume $F_0+F_1+F_2+\dots+F_k=F_{k+2}-1,k=0,1,2,\dots,n$\\
\textbf{Inductive Step:} Must prove $F_0+F_1+F_2+\dots+F_{n+1}=F_{(n+1)+2}-1$\\
\textbf{Prove:}\\
\begin{align*}
    F_0+F_1+F_2+\dots+F_{n+1} & = (F_0+F_1+F_2+\dots+F_n)+F_{n+1}\\
    & = (F_{n+2}-1)+F_{n+1} & \hfill \text{IH} \\
    & = F_{n+2}+F_{n+1}-1 & \hfill \text{} \\
    & = F_{n+3}-1 & \hfill \text{Def. Fibonacci number} \\
    & = F_{(n+1)+2}-1 & \hfill \text{} \\
\end{align*}
\textbf{Conclusion: }$F_0+F_1+F_2+\dots+F_n=F_{n+2}-1,n\ge0 $

\item[18] Consider the permutations of $1,2,3,4$. The permutation $1432$, for instance, is said to have one \textit{ascent} $14$.
\begin{enumerate}[label=\alph*)]
  \item
    $k=0: 321 \hspace{3mm} (1)$\\
    $k=1: 132, 213, 312, 231 \hspace{3mm} (4)$\\
    $k=2: 123 \hspace{3mm} (1)$
  \item
    $k=0: 4321 \hspace{3mm} (1)$\\
    $k=1: 1432, 4132, 3142, 2143, 3214, 4213, 4312, 4231, 2431, 3421, 3241 \hspace{3mm} (11)$\\
    $k=2: 1243, 1324, 1342, 1423, 2134, 3124, 4123, 2314, 3412, 2413, 2341 \hspace{3mm} (11)$\\
    $k=3: 1234 \hspace{3mm} (1)$
  \item
  $6-4=2$
  \item
  $m-1-k$
  \item
  $5: 912436587,192436587,129436587,124396587,124365987$\\
  $4: 124936587,124369587,124365897,124365879$\\
  \item
  $\pi_{m,k}=(m-k)\pi_{m-1,k-1}+(k+1)\pi_{m-1,k}$
\end{enumerate}
\end{description}

\end{document}