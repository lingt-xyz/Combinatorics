\documentclass[a4paper]{article}

\usepackage{fullpage} % Package to use full page
\usepackage{parskip} % Package to tweak paragraph skipping
\usepackage{tikz} % Package for drawing
\usepackage{amsmath}
\usepackage{hyperref}
\usepackage{amssymb}

\usepackage{enumitem}

\title{Chapter 9 Generating Functions}
\author{Ling Tan}
\date{2018/09}

\begin{document}

\maketitle

\section*{8.1 The Principle of Inclusion and Exclusion}
\color{red}e.g.: \color{black} Let $S$ be the set of 68 students in COMP339.\\
\begin{itemize}
    \item Condition $C_1$: 60 students from CS
    \item Condition $C_2$: 10 students from MATH
    \item $|S|=N=68, N(C_1)=60, N(C_2)=10, N(C_1C_2)=6$
\end{itemize}
\begin{align*}
N(\bar{C_1})& =N-N(C_1)=68-60=8\\
N(\bar{C_2})& =N-N(C_2)=68-10=58\\
N(C_1\bar{C_2})& =N(C_1)-N(C_1C_2)=60-6=54\\
N(\bar{C_1}{C_2})& =N(C_2)-N(C_1C_2)=10-6=4\\
N(\bar{C_1}\bar{C_2})& =N(\bar C_1)-N(\bar C_1C_2)\\
&=\bigl(N-N(C_1)\bigr)-\bigl(N(C_2)-N(C_1C_2)\bigr)\\
&=N-N(C_1)-N(C_2)+N(C_1C_2)\\
&=68-60-10+6=4\\
\end{align*}
$\Rightarrow N(\bar{C_1}\bar{C_2}\bar{C_3})=N-N(C_1)-N(C_2)-N(C_3)+N(C_1C_2)+N(C_2C_3)+N(C_1C_3)-N(C_1C_2C_3)$

\color{green}Theorem 8.1 \color{black} The Principle of Inclusion and Exclusion. Consider a set $S$, with $|S|=N$, and conditions $c_i,1\leq i\leq t$, each of which may be satisfied by some of the elements of $S$. The number of elements of $S$ that satisfy none of the conditions $c_i,1\leq i\leq t$, is 
denoted by $\bar {N}=N(\bar { c_1} \bar { c_2} \bar { c_3} \dots \bar { c_t})$ where
\begin{align*}
\bar {N} = & N\\
& -[N(c_1)+N(c_2)+N(c_3)+\dots + N(c_t)] \\
& +  [N(c_1 c_2)+N(c_1 c_3)+\dots+N(c_1 c_t)+N(c_2 c_3+\dots+N(c_{t-1} c_t))]\\ 
& -  [N(c_1 c_2 c_3)+N(c_1 c_2 c_4)+\dots+N(c_1 c_2 c_t)+N(c_1 c_3 c_4)+\dots+N(c_1 c_3 c_t)+\dots +N(c_{t-2} c_{t-1} c_t)]\\ 
& + \\
& \dots \\
& + (-1)^t N(c_1 c_2 c_3 \dots c_t)
\end{align*}
or
$$
\bar {N} = N - \sum_{1\leq i \leq t} N(c_i) + \sum_{1\leq i < j \leq t} N(c_i c_j) - \sum_{1\leq i < j < k \leq t} N(c_i c_j c_k) +  \dots + (-1)^t N(c_1 c_2 c_3 \dots c_t)
$$

\color{green}Corollary 8.1 \color{black} Under the hypotheses of Theorem 8.1, the number of elements in $S$ that satisfy at least one of the condition $c_i,1\leq i\leq t$, is given by $N(c_1\text{ or }c_2\text{ or }\dots\text{ or } c_t)=N-\bar{N}$.

\section*{8.2 Generalization of the Principle}
\color{green}Theorem 8.2 \color{black} Under the hypotheses of Theorem 8.1, for each $1\leq m\leq t$, the number of elements in $S$ that satisfy exactly $m$ of the conditions $c_1, c_2, \dots, c_t$ is given by
$$
E_m=S_m-\binom{m+1}{1}S_{m+1}+\binom{m+2}{2}S_{m+2}-\dots +(-1)^{t-m}\binom{t}{t-m}S_t
$$
If $m=0$, we obtain Theorem 8.1

\color{green}Corollary 8.2 \color{black} $L_m=S_m-\binom{m}{m-1}S_{m+1}+\binom{m+1}{m-1}S_{m+2}-\dots +(-1)^{t-m}\binom{t-1}{m-1}S_t$

\end{document}