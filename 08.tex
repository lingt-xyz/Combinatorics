\documentclass[a4paper]{article}

\usepackage{fullpage} % Package to use full page
\usepackage{parskip} % Package to tweak paragraph skipping
\usepackage{tikz} % Package for drawing
\usepackage{amsmath}
\usepackage{hyperref}
\usepackage{amssymb}

\usepackage{enumitem}

\title{8 The Principle of Inclusion and Exclusion}
\author{Ling Tan}
\date{2018/09}

\begin{document}

\maketitle

\section*{8.1 The Principle of Inclusion and Exclusion}

\textcolor{blue}{Theorem 8.1}: The Principle of Inclusion and Exclusion. Consider a set $S$, with $|S|=N$, and conditions $c_i,1\leq i\leq t$, each of which may be satisfied by some of the elements of $S$. The number of elements of $S$ that satisfy none of the conditions $c_i,1\leq i\leq t$, is 
denoted by $\bar {N}=N(\bar { c_1} \bar { c_2} \bar { c_3} \dots \bar { c_t})$ where
\begin{align*}
\bar {N} = & N\\
& -[N(c_1)+N(c_2)+N(c_3)+\dots + N(c_t)] \\
& +  [N(c_1 c_2)+N(c_1 c_3)+\dots+N(c_1 c_t)+N(c_2 c_3+\dots+N(c_{t-1} c_t))]\\ 
& -  [N(c_1 c_2 c_3)+N(c_1 c_2 c_4)+\dots+N(c_1 c_2 c_t)+N(c_1 c_3 c_4)+\dots+N(c_1 c_3 c_t)+\dots +N(c_{t-2} c_{t-1} c_t)]\\ 
& + \\
& \dots \\
& + (-1)^t N(c_1 c_2 c_3 \dots c_t)
\end{align*}
or
$$
\bar {N} = N - \sum_{1\leq i \leq t} N(c_i) + \sum_{1\leq i < j \leq t} N(c_i c_j) - \sum_{1\leq i < j < k \leq t} N(c_i c_j c_k) +  \dots + (-1)^t N(c_1 c_2 c_3 \dots c_t)
$$

\textcolor{blue}{Corollary 8.1}: Under the hypotheses of Theorem 8.1, the number of elements in $S$ that satisfy at least one of the condition $c_i,1\leq i\leq t$, is given by $N(c_1\text{ or }c_2\text{ or }\dots\text{ or } c_t)=N-\bar{N}$.

\subsection*{Examples}
\begin{enumerate}
    \item Let $S$ be the set of 68 students in COMP339.
    \begin{itemize}
        \item Condition $C_1$: 60 students from CS
        \item Condition $C_2$: 10 students from MATH
        \item $|S|=N=68, N(C_1)=60, N(C_2)=10, N(C_1C_2)=6$
    \end{itemize}
    \begin{align*}
    N(\bar{C_1})& =N-N(C_1)=68-60=8\\
    N(\bar{C_2})& =N-N(C_2)=68-10=58\\
    N(C_1\bar{C_2})& =N(C_1)-N(C_1C_2)=60-6=54\\
    N(\bar{C_1}{C_2})& =N(C_2)-N(C_1C_2)=10-6=4\\
    N(\bar{C_1}\bar{C_2})& =N(\bar C_1)-N(\bar C_1C_2)\\
    &=\bigl(N-N(C_1)\bigr)-\bigl(N(C_2)-N(C_1C_2)\bigr)\\
    &=N-N(C_1)-N(C_2)+N(C_1C_2)\\
    &=68-60-10+6=4
    \end{align*}
    $\Rightarrow N(\bar{C_1}\bar{C_2}\bar{C_3})=N-N(C_1)-N(C_2)-N(C_3)+N(C_1C_2)+N(C_2C_3)+N(C_1C_3)-N(C_1C_2C_3)$
    \item Find the number of positive integers $n$ where $1\leq n\leq 2000$ and $n$ cannot be divided by $2,3, 5$ or $7$.\\
    $N(C_1)=\lfloor2000/2\rfloor=1000, N(C_2)=\lfloor2000/3\rfloor=666,\\
    N(C_3)=\lfloor2000/5\rfloor=400,N(C_4)=\lfloor2000/7\rfloor=285.\\
    N(C_1C_2)=\lfloor2000/6\rfloor=333,N(C_1C_3)=\lfloor2000/10\rfloor=200, N(C_1C_4)=\lfloor2000/14\rfloor=142,\\
    N(C_2C_3)=\lfloor2000/15\rfloor=133,N(C_2C_4)=\lfloor2000/21\rfloor=95,\\
    N(C_3C_4)=\lfloor2000/35\rfloor=57.\\
    N(C_1C_2C_3)=\lfloor2000/30\rfloor=66,\\
    N(C_1C_2C_4)=\lfloor2000/42\rfloor=47,\\
    N(C_1C_3C_4)=\lfloor2000/70\rfloor=28,\\
    N(C_2C_3C_4)=\lfloor2000/105\rfloor=19,\\
    N(C_1C_2C_3C_4)=\lfloor2000/210\rfloor=9,\\
    2000-1000-666-400-285+333+200+142+133+95+57-66-47-28-19+9=458$
    \item Twelve people arrive at a movie theatre and line up to buy tickets (one ticket each). Tickets cost \$5 each, and 6 of the people want to pay with a \$5 bill and the other 6 with a \$10. Unfortunately, the box office clerk does not have any change. In how many ways can the people line up so that the clerk can give correct change to everyone?\\
    $\frac{1}{7}\binom{12}{6}\times 6!\times 6!$
    \item Solutions of $x_1+x_2+x_3+x_4\leq 18, 0\leq x_i\leq 7.$
    \begin{itemize}
        \item $|S|=N=S_0=\binom{4+18-1}{18}=\binom{21}{18}$
        \item $c_i: x_i>7\Rightarrow$ Want to know $N(\bar{c_1}\bar{c_2}\bar{c_3}\bar{c_4})$
        \item For $c_1$, consider $x_1+x_2+x_3+x_4=10\Rightarrow N(c_1)=\binom{4+10-1}{10}=\binom{13}{10}$
        \item $N(c_1)=N(c_2)=N(c_3)=N(c_4)$
        \item $N(c_1c_2)$, consider $x_1+x_2+x_3+x_4=2\Rightarrow N(c_1)=\binom{4+2-1}{2}=\binom{5}{2}$
        \item $N(c_1c_2c_3)=0$
        \item In total: $N(\bar{c_1}\bar{c_2}\bar{c_3}\bar{c_4})=S_0-S_1+S_2-S_3+S_4=\binom{21}{18}-\binom{4}{1}\binom{13}{10}+\binom{4}{2}\binom{5}{2}-0+0=246$
    \end{itemize}
    \item Example 8.9: Six married couples are to be seated at a circular table. In how many ways can they arrange themselves so that so wife sits next to her husband?
    \begin{itemize}
        \item $c_i$: the condition where a seating arrangement has couple $i$ seated next to each other.
        \item $N(c_1)=2\times (11-1)!$, because for a couple, husband and wife can switch.
        \item $N(c_1c_2)=2^2 \times (10-1)!$
        \item $N(c_1c_2c_3)=2^3 \times (9-1)!$
        \item $N(c_1c_2c_3c_4)=2^4 \times (8-1)!$
        \item $N(c_1c_2c_3c_4c_5)=2^5 \times (7-1)!$
        \item $N(c_1c_2c_3c_4c_5c_6)=2^6 \times (6-1)!$
        \item In total:\\
        \begin{align*}
            N(\bar{c_1}\bar{c_2}\bar{c_3}\bar{c_4}\bar{c_5}\bar{c_6})&=S_0-S_1+S_2-S_3+S_4-S_5+S_6\\
            &=(12-1)!\\
            &-\binom{6}{1}2(11-1)!+\binom{6}{2}2^2(10-1)!-\binom{6}{3}2^3(9-1)!\\
            &+\binom{6}{4}2^4(8-1)!-\binom{6}{5}2^5(7-1)!+\binom{6}{6}2^6(6-1)!
        \end{align*}
    \end{itemize}
    \item 17: In how many ways can three $x$’s, three $y$’s and three $z$’s be arranged so that no consecutive triple of the same letter appears?\\
    Let $c_1$: the three $x$'s are together; $c_2$: $y$'s; $c_3$: $z$'s.\\
    $N=\frac{9!}{3!3!3!},N(c_1)=N(c_2)=N(c_3)=\frac{7!}{3!3!},N(c_ic_j)=\frac{5!}{3!},N(c_1c_2c_3)=3!$\\
    $N(\bar{c_1}\bar{c_2}\bar{c_3})=S_0-S_1+S_2-S_3$
\end{enumerate}
\section*{8.2 Generalization of the Principle}
\textcolor{blue}{Theorem 8.2}: Under the hypotheses of Theorem 8.1, for each $1\leq m\leq t$, the number of elements in $S$ that satisfy exactly $m$ of the conditions $c_1, c_2, \dots, c_t$ is given by
$$
E_m=S_m-\binom{m+1}{1}S_{m+1}+\binom{m+2}{2}S_{m+2}-\dots +(-1)^{t-m}\binom{t}{t-m}S_t
$$
Proof: Arguing as in Theorem 8.1, let $x\in S$ and consider the following three cases.
\begin{enumerate}
    \item When $x$
    \item When $x$
    \item Suppose $x$
\end{enumerate}

If $m=0$, we obtain Theorem 8.1\\
\textcolor{blue}{Corollary 8.2}:  $L_m=S_m-\binom{m}{m-1}S_{m+1}+\binom{m+1}{m-1}S_{m+2}-\dots +(-1)^{t-m}\binom{t-1}{m-1}S_t$
\subsection*{Examples}
\begin{enumerate}
    \item 5: In how many ways can one distribute ten distinct prizes among four students with exactly two students getting nothing? How many ways have at least two students getting nothing?\\
    Here $A=\{1,2,3,\ldots, 10\},B=\{1,2,3,4\}$
    \item 8.2 Q2
\end{enumerate}
\end{document}