\documentclass[a4paper]{article}

\usepackage{fullpage} % Package to use full page
\usepackage{parskip} % Package to tweak paragraph skipping
\usepackage{tikz} % Package for drawing
\usepackage{amsmath}
\usepackage{hyperref}
\usepackage{amssymb}
\usepackage{enumitem}

\title{COMP 339 Assignment 2}
\author{Ling Tan}
\date{2018/09/27}

\begin{document}

\maketitle

\section{Exercises 8.1 (p. 396)}
\subsection{6: Integer solutions to $x_1+x_2+x_3+x_4=19$}
\begin{itemize}
    \item $0\leq x_i$ for all $1\leq i\leq 4$\\
    $$|S| = N =S_0=\binom{19+4-1}{19}=\binom{22}{19}$$
    \item $0\leq x_i < 8$ for all $1\leq i\leq 4$\\
    $$\text{Condition }c_i\text{ is }x_i>=8$$
    
    \item $0\leq x_1 \leq 5, 0\leq x_2 \leq 6, 3\leq x_3 \leq 7, 3\leq x_4 \leq 8$
\end{itemize}
\subsection{10: $x_1+x_2+\dots+x_{12}=200,10\leq x_i\leq25$}
\subsection{20}

\section{Exercises 8.2 (p. 401)}
\subsection{2}
\subsection{6}
\newpage
\section{Exercises 9.1 (p. 417) }
\subsection{2}
\begin{enumerate}[label=(\alph*)]
    \item $(1+x+x^2+x^3+\cdots)^35$
    \item $(x+x^2+x^3+\cdots)^35$
    \item $(x^2+x^3+\cdots)^35$
    \item $(x^{10}+x^{11}+x^{12}+\cdots)(1+x+x^2+x^3+\cdots)^4$
    \item $(x^{10}+x^{11}+x^{12}+\cdots)^2(1+x+x^2+x^3+\cdots)^3$
\end{enumerate}
    
\section{Exercises 9.2 (p. 431) }
\subsection{20}
\begin{enumerate}[label=(\alph*)]
    \item Palindrome of 11\\
    with 1: $11-1-1=9, 2^{\lfloor9/2\rfloor}=2^4=16$\\
    with 2: $11-2-2=7, 2^{\lfloor7/2\rfloor}=2^3=8$\\
    with 3: $11-3-3=5, 2^{\lfloor5/2\rfloor}=2^2=4$\\
    with 4: $11-4-4=3, 2^{\lfloor3/2\rfloor}=2^1=2$
    \item Palindrome of 12\\
    with 1: $12-1-1=10, 2^5=32$\\
    with 2: $12-2-2=8, 2^4=16$\\
    with 3: $12-3-3=6, 2^3=8$\\
    with 4: $12-4-4=4, 2^2=4$
\end{enumerate}

\subsection{30}
If $\{a_1,a_2,a_3,\ldots\}$ is a solution $\Rightarrow a_2-a_1>1,a_3-a_2>1,\ldots$\\
Let $c_2=a_2-a_1, c_3=a_3-a_2, \ldots$ and $c_1=a_1-1, c_8=50-a_7$
\begin{align*}
    \Rightarrow&c_2,c_3\cdots+c_7\geq2, c_1,c_8\geq0\\
    \Rightarrow&c_1+c_2+\cdots+c_8=50-1=49\\
    \text{That is }&(1+x+x^2+\cdot)^2(x^2+x^3+\cdots)^6\\
    =&\bigl(\frac{1}{1-x}\bigr)^2\Bigl(x^2\cdot\bigl(\frac{1}{1-x}\bigr)^6\Bigr)\\
    =&\frac{x^{12}}{(1-x)^8}\\
    \text{Coefficient is }&c_1'+c_2'+\cdots+c_8'=49-2\cdot6=37\text{ where }c_0,c_1,\ldots,c_8\geq 0\\
    \Rightarrow&\binom{-8}{37}(-1)^{37}=\binom{8+37-1}{37}\binom{44}{37}
\end{align*}
\section{Exercises 9.3 (p. 435)  }
\subsection{4}
\subsection{6}

\section{Exercises 9.4 (p. 439)  }
\subsection{2}
\begin{enumerate}[label=(\alph*)]
    \item $3,3^2,3^3,\ldots$
    \item $6\cdot 5^n - 3\cdot 2^n$
    \item $1, 1, 1,\ldots$
    \item $1,9,14,\ldots$
    \item $0!, 1!, 2!, \ldots$
    \item $3n!\cdot 2^n + 1$
\end{enumerate}

\subsection{6}

\section{Exercises 10.4 (p. 487) }
\subsection{1}
Let $f(x)$ denote $\sum_{n=0}^{\infty}{a_nx^n}$
\begin{align*}
\Rightarrow&\sum_{n=0}^{\infty}{a_{n+1}x^{n+1}} - \sum_{n=0}^{\infty}{a_nx^n} = \sum_{n=0}^{\infty}{3_nx^{n+1}}\\
\Rightarrow& f(x)-a_0-xf(x)=x\sum_{n=0}^{\infty}{3_nx^{n}}=\frac{x}{1-3x}\\
\Rightarrow& f(x)-1-xf(x)=\frac{x}{1-3x}\\
\Rightarrow&f(x)=\frac{1}{2}\frac{1}{1-x}+\frac{1}{2}\frac{1}{1-3x}\\
\Rightarrow&a_n=\frac{1}{2}(1+3^n)
\end{align*}
\subsection{3}

\section{Prove that if $n$ and $k$ are integers with $1\leq k \leq n$ then}
$$
k\cdot \binom{n}{k}=n\cdot \binom{n-1}{k-1}
$$

\subsection{By using a combinatorial proof.}
\begin{itemize}
    \item $k\cdot \binom{n}{k}\equiv$ number of the ways that 
        \begin{itemize}
            \item choose $k$ students from $n$ to form a committee: $\binom{n}{k}$
            \item then choose $1$ student from this committee to be a chairman (chairwoman): $\binom{k}{1}=k$
            \item As a result, we have formed a committee of $k$ students with $1$ chairman (chairwoman).
        \end{itemize}
    \item $n\cdot \binom{n-1}{k-1}\equiv$ number of the ways that 
        \begin{itemize}
            \item choose $1$ student from $n$ to be a chairman (chairwoman) of a committee: $\binom{n}{1}=n$
            \item then choose $k-1$ students from the rest $n-1$ students: $\binom{n-1}{k-1}$
            \item As a result, we have formed a committee of $k$ students with $1$ chairman (chairwoman).
        \end{itemize}
\end{itemize}

\subsection{By using an algebraic proof.}
\begin{align*}
    k\cdot \binom{n}{k} & = k\cdot \frac{n!}{k!(n-k)!}\\
    & = \frac{k}{k!}\cdot \frac{n!}{(n-k)!}\\
    & = n \cdot \frac{1}{(k-1)!}\cdot \frac{(n-1)!}{(n-k)!}\\
    & = n \cdot \frac{(n-1)!}{(k-1)!(n-k)!}\\
    & = n \cdot \frac{(n-1)!}{(k-1)!\bigl((n-1)-(k-1)\bigr)!}\\
    & = n\cdot \binom{n-1}{k-1}\\
\end{align*}

\section{Give a combinatorial proof that}
$$
\sum_{k=1}^n k\cdot \binom{n}{k} = n \cdot 2^{n-1}
$$
\begin{itemize}
    \item $\sum_{k=1}^n k\cdot \binom{n}{k}\equiv$ number of the ways that 
        \begin{itemize}
            \item choose any number of students ($\geq1$) to form a committee: $\binom{n}{k}, k=1,2,\dots,n$
            \item then choose $1$ student from that committee to be a chairman (chairwoman): $\binom{k}{1}=k, k=1,2,\dots,n$
            \item for every possible size of committee, we have $k\cdot\binom{n}{k}$ ways, where $k=1,2,\dots,n$
            \item As a result, we have formed a committee of any number of students with $1$ chairman (chairwoman).
        \end{itemize}
    \item $n \cdot 2^{n-1}\equiv$ number of the ways that 
        \begin{itemize}
            \item choose $1$ student from $n$ to be a chairman (chairwoman) of a committee: $\binom{n}{1}=n$
            \item then choose the rest members of that committee. Every student can either be chosen or not be chosen: $2^{n-1}$
            \item As a result, we have formed a committee of any possible size with $1$ chairman (chairwoman).
        \end{itemize}
\end{itemize}

\end{document}

