\documentclass[a4paper]{article}

\usepackage{fullpage} % Package to use full page
\usepackage{parskip} % Package to tweak paragraph skipping
\usepackage{tikz} % Package for drawing
\usepackage{amsmath}
\usepackage{hyperref}
\usepackage{amssymb}
\usepackage{enumitem}

\title{COMP 339 Assignment 2}
\author{Ling Tan}
\date{2018/09/27}

\begin{document}

\maketitle

\section*{Exercises 8.1 (p. 396)}
\begin{description}
\item[6] Integer solution to $x_1+x_2+x_3+x_4=19$.
\begin{enumerate}[label=\alph*)]
    \item $0\leq x_i$ for all $1\leq i\leq 4$\\
    $$|S| = N =S_0=\binom{19+4-1}{19}=\binom{22}{19}$$
    \item $0\leq x_i < 8$ for all $1\leq i\leq 4$\\
    $$\text{Condition }c_i\text{ is }x_i>=8$$
    \item $0\leq x_1 \leq 5, 0\leq x_2 \leq 6, 3\leq x_3 \leq 7, 3\leq x_4 \leq 8$
\end{enumerate}
\item[10] $x_1+x_2+\dots+x_{12}=200,10\leq x_i\leq25,x_i$ is a multiple of $5$.
\item[20] At a 12-week conference in mathematics, Sharon met seven of her friends from college. During the conference she met each friend at lunch 35 times, every pair of them 16 time, every trio eight time, every foursome four times, each set of five twice, and each set of six once, but never all seven at one. If she had lunch every day during the 84 days of the conference, did she ever have lunch along?\\
\begin{equation*}
    84-\binom{7}{1}(35)+\binom{7}{2}(16)-\binom{7}{3}(8)+\binom{7}{4}(4)-\binom{7}{5}(2)+\binom{7}{6}(1)-\binom{7}{7}=0
\end{equation*}
\end{description}

\section*{Exercises 8.2 (p. 401)}
\begin{description}
\item[2] Arrange letter ARRANGEMENT
\begin{enumerate}[label=\alph*)]
    \item There are exactly two pair of consecutive identical letters? at least two pairs of consecutive identical letters?
    \item Answer (a), replacing two with three.
\end{enumerate}
\item[6] Zelma is having a luncheon for herself and nine of the women in her tennis league. On the morning of the luncheon she places name cards at the ten places at her table and then leaves to run a last-minute errand. Her husband, Herbert, comes from his morning tennis match and unfortunately leaves the back door open. A gust of wind scatters the ten name cards. In how many ways can Herbert replace the ten cards at the places at the table so that exactly four of the ten women will be seated where Zelma had wanted them? In how many ways will at least four of them be seated where they were supposed to be?
\end{description}

\section*{Exercises 9.1 (p. 417) }
\begin{description}
\item [2]
\begin{enumerate}[label=(\alph*)]
    \item $(1+x+x^2+x^3+\cdots x^{35})^5$
    \item $(x+x^2+x^3+\cdots)^35$
    \item $(x^2+x^3+\cdots)^35$
    \item $(x^{10}+x^{11}+x^{12}+\cdots)(1+x+x^2+x^3+\cdots)^4$
    \item $(x^{10}+x^{11}+x^{12}+\cdots)^2(1+x+x^2+x^3+\cdots)^3$
\end{enumerate}
\end{description}
    
\section*{Exercises 9.2 (p. 431) }
\begin{description}
\item[20] Palindrome
\begin{enumerate}[label=(\alph*)]
    \item Palindrome of 11\\
    with 1: $11-1-1=9, 2^{\lfloor9/2\rfloor}=2^4=16$\\
    with 2: $11-2-2=7, 2^{\lfloor7/2\rfloor}=2^3=8$\\
    with 3: $11-3-3=5, 2^{\lfloor5/2\rfloor}=2^2=4$\\
    with 4: $11-4-4=3, 2^{\lfloor3/2\rfloor}=2^1=2$
    \item Palindrome of 12\\
    with 1: $12-1-1=10, 2^5=32$\\
    with 2: $12-2-2=8, 2^4=16$\\
    with 3: $12-3-3=6, 2^3=8$\\
    with 4: $12-4-4=4, 2^2=4$
\end{enumerate}
\item [30]
If $\{a_1,a_2,a_3,\ldots\}$ is a solution $\Rightarrow a_2-a_1>1,a_3-a_2>1,\ldots$\\
Let $c_2=a_2-a_1, c_3=a_3-a_2, \ldots$ and $c_1=a_1-1, c_8=50-a_7$
\begin{align*}
    \Rightarrow&c_2,c_3\cdots+c_7\geq2, c_1,c_8\geq0\\
    \Rightarrow&c_1+c_2+\cdots+c_8=50-1=49\\
    \text{That is }&(1+x+x^2+\cdot)^2(x^2+x^3+\cdots)^6\\
    =&\bigl(\frac{1}{1-x}\bigr)^2\Bigl(x^2\cdot\bigl(\frac{1}{1-x}\bigr)^6\Bigr)\\
    =&\frac{x^{12}}{(1-x)^8}\\
    \text{Coefficient is }&c_1'+c_2'+\cdots+c_8'=49-2\cdot6=37\text{ where }c_0,c_1,\ldots,c_8\geq 0\\
    \Rightarrow&\binom{-8}{37}(-1)^{37}=\binom{8+37-1}{37}\binom{44}{37}
\end{align*}
\end{description}
\section*{Exercises 9.3 (p. 435)  }
\begin{description}
\item[4] Find the generating function for the number of integer solutions of
\begin{enumerate}[label=\alph*)]
    \item $2w+3x+5y+7z=n,\quad 0\leq w,x,y,z$\\
    $f(x)=(1+x^2+x^4+\cdots)(1+x^3+x^6+\cdots)(1+x^5+x^{10}+\cdots)(1+x^7+x^{14}+\cdots)$
    \item $2w+3x+5y+7z=n,\quad 0\leq w, 4\leq x,y, 5\leq z$\\
    $f(x)=(1+x^2+x^4+\cdots)(x^{12}+x^{15}+x^{18}+\cdots)(x^{20}+x^{25}+x^{30}+\cdots)(x^{35}+x^{42}+x^{49}+\cdots)$
\end{enumerate}
\item[6] What is the generating function for the number of partitions of $n\in\mathbb{N}$ into summands that
\begin{enumerate}[label=(\alph*)]
    \item Cannot occur more than five times?
    \begin{itemize}
        \item For $1$: one time is $1$, two times is $2$, $\cdots$, five times is $5$.
        \item For $2$: one time is $2$, two times is $4$, $\cdots$, five times is $10$
        \item $\cdots$
        \item For $i$: one time is $i$, two times is $2i$, $\cdots$, five times is $5i$
    \end{itemize}
    \begin{align*}
    f(x)&=(1+x+x^2+\cdots +x^5)(1+x^2+x^4+\cdots +x^{10})\cdots\\
    &=\Pi_{i=1}^{\infty}{(1+x^i+x^{2i}+\cdots+x^{5i})}\\
    &=\Pi_{i=1}^{\infty}{\frac{1}{1-x^2}}\\
    \text{Coefficient of }&x^n
    \end{align*}
    \item Cannot exceed 12 and cannot occur more than five times?
    $x_1+x_2+\cdots+x_i\leq 12,i\leq 5$
\end{enumerate}
\end{description}

\section*{Exercises 9.4 (p. 439)  }
\begin{description}
\item[2]
\item[6]
\begin{enumerate}[label=(\alph*)]
    \item $3,3^2,3^3,\ldots$
    \item $6\cdot 5^n - 3\cdot 2^n$
    \item $1, 1, 1,\ldots$
    \item $1,9,14,\ldots$
    \item $0!, 1!, 2!, \ldots$
    \item $3n!\cdot 2^n + 1$
\end{enumerate}
\end{description}


\section*{Exercises 10.4 (p. 487) }
\begin{description}
\item[1] Solve the following recurrence relations by the method of generating functions.
\begin{enumerate}[label=\alph*)]
    \item $a_{n+1}-a_n=3^n,n\geq 0, a_0=1$\\
    Let $f(x)$ denote $\sum_{n=0}^{\infty}{a_nx^n}$
    \begin{align*}
    \Rightarrow&\sum_{n=0}^{\infty}{a_{n+1}x^{n+1}} - \sum_{n=0}^{\infty}{a_nx^n} = \sum_{n=0}^{\infty}{3_nx^{n+1}}\\
    \Rightarrow& f(x)-a_0-xf(x)=x\sum_{n=0}^{\infty}{3_nx^{n}}=\frac{x}{1-3x}\\
    \Rightarrow& f(x)-1-xf(x)=\frac{x}{1-3x}\\
    \Rightarrow&f(x)=\frac{1}{2}\frac{1}{1-x}+\frac{1}{2}\frac{1}{1-3x}\\
    \Rightarrow&a_n=\frac{1}{2}(1+3^n)
    \end{align*}
    \item $a_{n+1}-a_n=n^2,n\geq 0,a_0=1$
    \item $a_{n+2}-3a_{n+1}+2a_n=0,n\geq 0,a_0=1,a_1=6$
    \item $a_{n+2}-2a_{n+1}+a_n=2^n,n\geq 0,a_0=1,a_1=2$
\end{enumerate}

\item[3] Solve the following systems of recurrence relations.
\begin{enumerate}[label=\alph*)]
    \item TODO
    \item TODO
\end{enumerate}
\end{description}

\section*{Prove}
If $n$ and $k$ are integers with $1\leq k \leq n$ then
$$
k\cdot \binom{n}{k}=n\cdot \binom{n-1}{k-1}
$$
\begin{description}
\item [a)]{By using a combinatorial proof.}
\begin{itemize}
    \item $k\cdot \binom{n}{k}\equiv$ number of the ways that 
        \begin{itemize}
            \item choose $k$ students from $n$ to form a committee: $\binom{n}{k}$
            \item then choose $1$ student from this committee to be a chairman (chairwoman): $\binom{k}{1}=k$
            \item As a result, we have formed a committee of $k$ students with $1$ chairman (chairwoman).
        \end{itemize}
    \item $n\cdot \binom{n-1}{k-1}\equiv$ number of the ways that 
        \begin{itemize}
            \item choose $1$ student from $n$ to be a chairman (chairwoman) of a committee: $\binom{n}{1}=n$
            \item then choose $k-1$ students from the rest $n-1$ students: $\binom{n-1}{k-1}$
            \item As a result, we have formed a committee of $k$ students with $1$ chairman (chairwoman).
        \end{itemize}
\end{itemize}

\item [b)]{By using an algebraic proof.}
\begin{align*}
    k\cdot \binom{n}{k} & = k\cdot \frac{n!}{k!(n-k)!}\\
    & = \frac{k}{k!}\cdot \frac{n!}{(n-k)!}\\
    & = n \cdot \frac{1}{(k-1)!}\cdot \frac{(n-1)!}{(n-k)!}\\
    & = n \cdot \frac{(n-1)!}{(k-1)!(n-k)!}\\
    & = n \cdot \frac{(n-1)!}{(k-1)!\bigl((n-1)-(k-1)\bigr)!}\\
    & = n\cdot \binom{n-1}{k-1}\\
\end{align*}
\end{description}
\section*{Prove}
Give a combinatorial proof that
$$
\sum_{k=1}^n k\cdot \binom{n}{k} = n \cdot 2^{n-1}
$$
\begin{itemize}
    \item $\sum_{k=1}^n k\cdot \binom{n}{k}\equiv$ number of the ways that 
        \begin{itemize}
            \item choose any number of students ($\geq1$) to form a committee: $\binom{n}{k}, k=1,2,\dots,n$
            \item then choose $1$ student from that committee to be a chairman (chairwoman): $\binom{k}{1}=k, k=1,2,\dots,n$
            \item for every possible size of committee, we have $k\cdot\binom{n}{k}$ ways, where $k=1,2,\dots,n$
            \item As a result, we have formed a committee of any number of students with $1$ chairman (chairwoman).
        \end{itemize}
    \item $n \cdot 2^{n-1}\equiv$ number of the ways that 
        \begin{itemize}
            \item choose $1$ student from $n$ to be a chairman (chairwoman) of a committee: $\binom{n}{1}=n$
            \item then choose the rest members of that committee. Every student can either be chosen or not be chosen: $2^{n-1}$
            \item As a result, we have formed a committee of any possible size with $1$ chairman (chairwoman).
        \end{itemize}
\end{itemize}

\end{document}

